\chapter{Le travail réalisé}

%
%
%
\section{État de l'existant}
Quand je suis arrivé au laboratoire \textit{CIRDLES}, le coeur du logiciel avait déja été réalisé par John Zeringue, mon partenaire principal pendant ce stage. Il y avait déjà un tableau où l'on pouvait entrer des données, et on pouvait déja génèrer un graphique avec les ellipses d'erreurs et une des lignes Concordia\footnote{On savait pas d'ailleurs à l'époque qu'il y avait plusieurs lignes Concordia. On appelait tout simplement celle qui était dans le logiciel ``La ligne concordia''. C'est bien plus tard dans le projet que l'on a découvert qu'il y en avait plusieurs.}.

Le moment où je suis arrivé coincide avec le moment où le Docteur Bowring nous a donné le cas d'utilisation primaire du chercheur Australien à implémenter. Le cas d'utilisation peut se résumer aux étapes suivantes :
\begin{enumerate}
\item Le geochronologiste a des données dans une feuille Excel qui représente des ratios et leur inexactitude. Il les copie.
\item Il entre dans notre logiciel et colle les donnée dans un tableau similaire.
\item Il demande à génèrer le schema, le logiciel le lui montre.
\item Il a ensuite la possibilité de l'exporter au format SVG
\end{enumerate}

Après avoir passé deux jours à me renseigner sur la géochronologie en lisant une publication du Docteur Bowring ou encore en regardant un DVD sur le sujet, %Cette phrase mérite peut-être sa propre section. Ou elle l'aura dans la quatrieme partie.
j'ai commencé avec John à implémenter ce cas d'utilisation.

%
%
%
\section{L'implementation du cas d'utilisation}
Il n'y a avait pas beaucoup de travail pour adapter le logiciel au cas d'utilisation. Pendant que John s'est occupé de l'import de données depuis \textit{Excel}, j'ai créé le \textit{ColumnSelectorView}.

%
%
\subsection{Le \textit{ColumnSelectorView}}
Dans la version d'alors du logiciel, le tableau n'avait que cinq colonnes, une pour chaque variables dont nous avions besoin pour tracer les ellipses d'erreurs. Nous avons décidé de permettre aux utilisateurs d'insérer autant de colonnes qu'ils souhaitaient depuis leur feuille de calcul Excel. Cela demandait cependant une étape supplémentaire : lier les colonnes aux variables. C'est exactement le travail du \textit{ColumnSelectorView}. 

%--------------INSERER ICI UNE CAPTURE D'ECRAN DU COLONNESELECTORVIEW. DON'T FORGET--------------

D'un point de vue graphique cette interface n'est pas très interessante. La façon dont je l'ai implementé sera par contre très interessante pour le chapitre quatre, qui portera sur \textit{ce que j'ai appris durant ce stage}. La première monture de cette classe ne comportait pas la logique pour de la génération du schema. Ma conception de cette classe était un petit utilitaire qui ne sert qu'à choisir les champs, la génération du schema devait se faire autre part. Cette classe ne contenait qu'un modele de \textit{listener}\footnote{Un listener est une interface (au sens programatique du terme) qui réagit aux signaux d'un objet en particulier.} qui contenait une méthode que l'on appelait pour gênerer le schema. Ce listener était implémenté dans la classe qui gêrait le tableau. Pour moi, cela fait sens que la génération d'un graphique avec les données d'un tableau soit ordonnées par celui-ci. Ce n'est pas faux en soit, mis il y a mieux...

%--------------INSERER ICI LE CODE DE L'INTERFACE YO--------------

%TODO : Quinze jours est faux, trouver le bon 
Quinze jours plus tard, le code de cette classe avait drastiquement changé. Je vais être honnête, je n'ai pas fais ces changements. John a commencé à me faire comprendre que ma façon de concevoir était très démodée en supprimant l'interface %Nom de l'interface
. Laissez moi vous expliquer plus en détails.

Nous utilisons dans le projet une librairie appelée \textit{ControlsFX}. Elle ajoute des éléments d'interface graphique qui ne sont pas dans la librairie originale, ainsi que la notion d'\textbf{Actions}. Une Action est une classe qui \textit{fait quelque chose en réaction à une demande de l'utilisateur}. Un de ses avantages est qu'elle est automatiquement transformée en élément d'interface par \textit{ControlsFX}. Exemple : une boite de dialogue a une série d'actions possibles en réaction à son affichage qu'elle tranforme en boutons.

John a tranformé ma classe qui était auparavant un simple élément affichable par \textit{JavaFX} en boite de dialogue de JavaFX. Elle était dorénavant mieux integrée stylistiquement à l'interface de l'application, et surtout, elle enlevait le \textit{listener} du tableau, et donnait à la logique de génération de tableau sa propre classe : une action. Bien plus propre.

Aujourd'hui cette Action est toujours une classe gigogne du \textit{ColumnSelectorView}, ce qui est selon toute l'équipe une place non idéale. Il faudra bientôt que l'on déplace toutes nos actions dans un package qui leur est dédié.

%
%
\subsection{Completion du convertisseur SVG}
Il ne restait alors plus qu'une seule tâche avant que l'implémentation du cas d'utilisation soit terminée : La conversion du graphique en un fichier SVG.

\begin{center}
Le SVG est un format graphique vectoriel. Un fichier au format SVG contient la définitions de diverse formes. Ces défintion comportent entre autres la positions de la position de la forme, la couleur de son tracé et de son remplissage, la largeur de son tracé \ldots.
\end{center}

\begin{center}
Un noeud (\textit{Node}) dans \textit{JavaFX} est la classe de base de n'importe quel élément affichable. La plupart du temps, on utilise des noeuds composés d'autre noeuds. Un simple bouton, par exemple, contient au moins un rectangle et un label\footnote{Nom d'un texte affiché à l'écran dans le jargon informatique}.
\end{center}

\begin{center}
Une scène (\textit{Scene}) est un ensemble de noeuds.
\end{center}

J'ai repris en charge l'écriture d'un convertisseur que John avait commencé à écrire. Ce convertisseur est conçu pour pouvoir transformer n'importe quel noeud en SVG. Pour cela il parcours tous les noeuds composés récursivement\footnote{La récursivité est un concept de l'informatique qui désigne la répétition d'une action entrainée par cette même action. Ici, l'action serait ``Parcourir les enfants''. Quand je \emph{parcours les enfants}, si je trouve un enfants qui en a d'autres, je \emph{parcours ses enfants}.}, et converti les noeuds de bases en éléments SVG. Nous nous en servons sur le noeud du graphique.

Avant que je travail dessus, voici comme le graphiques étaient rendus :
%----------------Image du rendu avant--------------

Le travail que j'ai du effectuer pour ce convertisseur est principalement un travail d'investigation. Il fallait que je trouve où et comment JavaFX définit les options grahiques des noeuds qu'il utilise au moment de rendre les noeuds.

Pour cela, j'ai tout d'abord regardé comment le noeud du graphique était composé. J'ai extrait la hierarchie des enfants du noeud du graphique :

\begin{verbatim}
ConcordiaChart@b721ec6[styleClass=chart]
   Label@58eeaa1f[styleClass=label chart-title]''
      Text[text="", x=0.0, y=0.0, alignment=LEFT, origin]
   Chart$1@5109fabb[styleClass=chart-content]
      Region@3078127b[styleClass=chart-plot-background]
      XYChart$1@206ec019
          Path[elements=[], fill=null, fillRule=NON_ZERO]
          Path[elements=[MoveTo[x=100.0, y=10.0], LineTo[x=1 
          Path[elements=[MoveTo[x=297.5, y=10.0], LineTo[x=2
          Path[elements=[MoveTo[x=100.0, y=646.5], LineTo[x 
          Line[startX=0.0, startY=0.0, endX=0.0, endY=0.0, s 
          Line[startX=0.0, startY=0.0, endX=0.0, endY=0.0, s
          Group@5867eb04[styleClass=plot-content]
              Group@789d6eba
                  Path[elements=[MoveTo[x=0.0, y=-57194.0], LineTo[x Invisible
              Group@400b7196[styleClass=error-ellipse]
                  Path[elements=[MoveTo[x=1023.0, y=62.0], CubicCurv
              Group@bddd852[styleClass=error-ellipse]
                  Path[elements=[MoveTo[x=798.0, y=206.0], CubicCurv
                  Circle[centerX=775.0, centerY=221.0, radius=3.0, f
              Group@27c9ed47[styleClass=error-ellipse]
                  Path[elements=[MoveTo[x=604.0, y=481.0], CubicCurv
                  Circle[centerX=583.0, centerY=497.0, radius=3.0, f
              Group@5c612db6[styleClass=error-ellipse]
                  Path[elements=[MoveTo[x=103.0, y=598.0], CubicCurv
                  Circle[centerX=83.0, centerY=615.0, radius=3.0, fi
              Group@1e4ce279[styleClass=error-ellipse]
                  Path[elements=[MoveTo[x=810.0, y=538.0], CubicCurv
                  Circle[centerX=789.0, centerY=557.0, radius=3.0, f
              Group@5eec36db[styleClass=error-ellipse]
                  Path[elements=[MoveTo[x=433.0, y=339.0], CubicCurv
                  Circle[centerX=411.0, centerY=352.0, radius=3.0, f
      NumberAxis@2b5f1854[styleClass=axis] Horizontal 
         Label@29a340ae[styleClass=label axis-label]'xxxPb/
              Text[text="xxxPb/xxxU", x=0.0, y=0.0, alignment=LE 207pb/235u
          Path[elements=[MoveTo[x=0.0, y=0.0], LineTo[x=0.0, 
          Path[elements=[MoveTo[x=40.0, y=1.0], LineTo[x=40. 
          Text[text="0.0720000", x=0.0, y=0.0, alignment=LEF
          Text[text="0.0720500", x=0.0, y=0.0, alignment=LEF
          Text[text="0.0721000", x=0.0, y=0.0, alignment=LEF
          Text[text="0.0721500", x=0.0, y=0.0, alignment=LEF
          Text[text="0.0722000", x=0.0, y=0.0, alignment=LEF
          Text[text="0.0722500", x=0.0, y=0.0, alignment=LEF
      NumberAxis@2650091e[styleClass=axis]
          Label@2ea49c54[styleClass=label axis-label]'xxxPb/
              Text[text="xxxPb/xxxU", x=0.0, y=0.0, alignment=LE Invisible
          Path[elements=[MoveTo[x=82.0, y=636.0], LineTo[x=9
          Path[elements=[MoveTo[x=85.0, y=679.0], LineTo[x=8
          Text[text="0.0000800", x=0.0, y=0.0, alignment=LEF
          Text[text="0.0001000", x=0.0, y=0.0, alignment=LEF
          Text[text="0.0001200", x=0.0, y=0.0, alignment=LEF
          Text[text="0.0001400", x=0.0, y=0.0, alignment=LEF
          Text[text="0.0001600", x=0.0, y=0.0, alignment=LEF
          Text[text="0.0001800", x=0.0, y=0.0, alignment=LEF
          Text[text="0.0002000", x=0.0, y=0.0, alignment=LEF
          Text[text="0.0002200", x=0.0, y=0.0, alignment=LEF
      Rectangle[x=0.0, y=0.0, width=0.0, height=0.0, fil Empty
\end{verbatim}
J'ai cherché à savoir, en comparant avec les éléments du SVG déjà génèré par le code existant à quel élément graphique correspondait chacuns de ces éléments. Pour une meilleurs de ce qui suit, une capture d'écran d'un graphique a été placé après cette liste. On voit donc que le \textit{ConcordiaChart} est composé de :
\begin{itemize}
\item Un objet \textit{Label} : le titre du graphique
\item Un objet \textit{Chart} avec un objet \textit{XYChart} contenant :
  \begin{itemize}
  \item Des objets \textit{Path} : Les lignes étendant les graduations des axes
  \item Un objet \textit{Group} : Les objets affichés sur le graphique
    \begin{itemize}
    \item La ligne de temps est l'objet \textit{Path} dans le second \textit{Group}
    \item Les autres \textit{Group} contiennent un \textit{Path}, l'Ellipse, et un \textit{Circle}, le point au milieu de l'ellipse.
    \end{itemize}
  \end{itemize}
\item Deux objets \textit{Axis}, contenant chacun :
  \begin{itemize}
  \item Un \textit{Label} : la légende de l'axe
  \item Deux éléments \textit{Path} : Les grandes et petites marques de graduation
  \item Des éléments \textit{Text} : Les textes des graduations
  \end{itemize}
\end{itemize}
%Image du graphique sous JavaFX

J'ai ensuite été chercher dans le moteur de rendu de \textit{JavaFX} la logique qui rend une scene à l'écran. Mon plan (diabolique s'il en est), était de voir où cette logique récupère les options graphiques des objets qu'elle doit dessiner. Si mon plan a failli, me coutant quelques jours, j'ai néanmoins appris énormément sur la façon dont \textit{JavaFX} fonctionne. 

Pour résumer, le code de JavaFX est separé en deux :
\begin{enumerate}
\item La partie publique est celle que l'on utilise pour construire une scene, et plus génénalement pour intéragir avec la librairie. Elle contient aussi les \textit{Property}, dont je parlerais plus tard. Elle est sous le package %TODO noter le package avec Verbatim
\item La partie privée (joyeusement fouillie et peu commentée) a différents rôles:%TODO noter le package avec verbatim
  \begin{enumerate}	
  \item Rendre des scenes (Moteur de rendu \textit{Prism}, sur lequel je me suis particulierement penché)
  \item Unifier l'interaction avec les systèmes de fenêtrage des différents OS avec lesquels \textit{JavaFX} est compatible. (Module \textit{Glassfish})
  \item Rendre différents type de médias
  \item Rendu du HTML complet, (Moteur basé sur \textit{Webkit})
  \item Marcher sur plusieurs threads
  \end{enumerate}
\end{enumerate}

Après la faillite de mon premier plan, j'ai procédé plus simplement. Je suis revenu aux objets de l'API publique, ceux que nous manipulons quand nous créons une scène et ait cherché les différentes options graphique en leur seins. J'ai reussi à trouver et à orienter correctement le nom des axes et à repositionner ceux-ci, j'ai re-aligné les graduations, repositionné et affiché des lignes et des textes dans le graph.

%
%
%
\section{Réponse aux attente de la communauté GitHub}
Le module SVG était fini. Le cas d'utilisation était implémenté. Nous étions près à montrer le fruit de notre travail aux chercheurs Australiens, ce que nous avons fait le Jeudi% verifier que c'est  bien un jeudi 
21 mai par visioconférence. Ils étaient enchantés du prototype.

Le docteur Bowring a envoyé un lien vers le dépot GitHub du projet, et une petite mais active communauté s'est formée autour du projet. Ses membres nous envoient encore à ce jours des rapports de bug, des suggestions de fonctions.

Cela a bien sur beaucoup changer la dynamique de travail. Nos journées étaient à présent partagées entre réponses à la communauté, corrections rapides de bugs et developpement de fonctions plus complexes. Nous avons appris à penser à court et long termes.

Je vais à présent parler des deux dernières fonctions complexes que j'ai conçu et codé les semaines restantes.
