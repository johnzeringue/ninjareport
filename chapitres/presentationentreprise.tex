\part{Présentation de l'entreprise}
\chapter{College of Charleston, l'organisation d'accueil}

Le College of Charleston, abrégé Cofc, fondé en 1770, situé dans la ville de Charleston, SC, USA, regroupe onze mille étudiants dans des programmes appartenant majoritairement au premier cycle d’études supérieur américain. Le champs des domaines enseignés est large, et va des arts aux sciences, en passant par le business, l'éducation ou encore les humanités.
 
Il est situé dans la ville de Charleston, ville de plus de 120 000 habitants situé au bord de la mer, lieu touristique reconnu pour sa valeur historique, sa vie culturelle et sportive très active, un climat plaisant et des conditions de vie plaisante.
Une des particularité du College de Charleston est qu’il forme ses étudiants aux arts libéraux.  La notion des arts liberaux nous vient de l’antiquité. Le sens commun de cette expression désigne l'intégralité des matières qu’un homme doit étudier afin de devenir un citoyen accompli. Plus spécifiquement, a liberal arts college forme ses étudiants à de multiples sujets, leur inculquant une culture générale solide, les incitant aux découverte et à l’exploration.

De nombreux ex-étudiants du cofc ont réalisé de grandes choses. Parmi eux, un acteur hollywoodien, un membre des New York Yankee, un musicien éléctronique reconnu, et un directeur des effets speciaux ayant reçu un oscar pour son travail sur Pirate des Caraïbe 3.

En plus d’enseigner, le college a en son sein de nombreux laboratoires et participe activement à la recherche scientifique.

\chapter{Le laboratoire CIRDLES, la cellule d’accueil}
J’ai effectué mon stage dans le laboratoire CIRDLES, partie du Department of Computer Science, lui-même partie de la School of Sciences and Mathematics.

The School of Sciences and Mathematics est une des six divisions du College. Cette école est considéré comme la plus performante dans la recherche et l’enseignement depuis de nombreuses années au niveau de l’état de Caroline du Sud. Elle profite de son remarquable environnement pour sortir ses étudiants des salles de classes afin de se rapprocher de leurs sujets d’études. Elle siêge dans un bâtiment neuf et en parfait état, le dernier ouvert par le Cofc. Suivant la mission du College de Charleston de former des hommes curieux et explorateurs, elle propose en plus des programmes normaux, d’autres mélangeant plusieurs de ses disciplines.

Le département informatique propose quatre diplômes du premier cycle ainsi qu’un diplôme du second cycle. Il est à dimension humaine, les professeurs connaissent et appellent leurs élèves par leur prénoms, sont à l’écoute et accessible aisément (Même spécialement parlant, leur bureaux se trouvant en face des salles de classes).
L’ambiance du département est exceptionnel. Les étudiants se connaissent, s’entraident, se retrouvent dans une salle d’étude en deux parties, une consacrée à la détente et l’autre au travail. Ils se regroupent au seins de plusieurs associations. Une des ces association, ACM, propose une conférence hebdomadaire donné par des professionnels de la sécurité informatique, une autre Gaming 101 organise une soirée jeux vidéos par semaine.

Le laboratoire CIRDLES développe des outils informatiques destiné aux geochronologistes du monde entier. Il est dirigé par Dr Jim Bowring. Le travail du laboratoire s’articule autour d’U-Pb-Redux et Tripoli, deux logiciels développé par Dr Bowring. Les membres du laboratoire, des étudiants rémunérés, se voient confier la réalisations de projets en rapports avec ces outils, comme la réalisation d’une applications Android permettant de consulter des fichiers issus d’U-Pb-Redux ou des tests. Dans certains autres cas, ils doivent commencer un logiciels de zéro pour satisfaire certains laboratoire.
Présentation des outils informatiques utilisés
J’utilise un iMac doté deux écrans tournant sous Mac OSX 10.9.2. La librairie que l’on développe est codée avec le langage Java, dans sa huitième version. Elle étends les fonctionnalités de JavaFX, une librairie graphique permettant l’affichage par exemple de fenêtres.
Plus tard dans le semestre, nous auront à mettre en place une communauté open source. Nous utiliseront pour cela probablement le site web Github. Il regroupant le code source de milliers d’application, proposant une plateforme, facilitant la collaboration entre developpeurs et la communications avec les utilisateurs. Pour chaque projets, GitHub propose un dépot Git pour centraliser le code de l’application, un wiki pour la documentation et un système de suivi de bogue.
