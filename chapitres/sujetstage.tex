\chapter{Le sujet du stage}
\section{Géochronologie} %"Un peu de Géochronologie..." ?
Le carbon 14 est une isotope populairement connu comme permetant de dater un objet. Il n'est pas seul. L'uranium et le plomb permettent aussi, dans des conditions particulières, de dater une couche du globe terrestre.

Quand un volcan entre en éruption, en plus de la lave expulsé de son cratère, celui-ci crache aussi des \textit{zircons}. Un zircon est un crystal qui, lorsqu'il est créé, ne comporte que de l'uranium. Il est important de noter que ces cristaux sont des capsules hermétiques : rien n'en sortira et n'y rentrera. Ces zircons atterissent dans la lave et se retrouve emprisonnés. 

Le temps passe, de nouvelles couches se forment, au dessus de la lave. L'uranium à l'interieur des zircons se décompose en plomb, à une vitesse bien définie et connue des chercheurs. 

Des millions d'années plus tard, quand les chercheurs se posent la question de l'âge d'une couche terrestre au pied d'un volcan, ils cassent sur place des énormes bloques de lave solidifés. Par un procédé complexe, ils en extraient ensuite les zircons. En mesurant le rapport entre l'uranium et le plomb présent dans les zircons, les chercheurs peuvent déterminer l'âge de la couche terrestre où ils ont été trouvés.

Le but du laboratoire \textit{CIRDLES} est d'aider les chercheurs dans cette tâche précise et complèxe.

\section{Le logiciel sur lequel j'ai travaillé : Topsoil}
Le logiciel sur lequel j'ai travaillé s'appelle \textit{Topsoil}. Il permet de produire ce genre de graphiques :
%--------------INSERER ICI GRAPHIQUE. DON'T FORGET--------------
Il remplace un autre logiciel nommé \textit{Isotop}.

\subsection{Les ellipses d'erreurs}
Une ellipse représentée sur ce shema montre la relation entre deux ratios entre des isotopes de plomb et d'uranium. Un isotope est pour faire simple, une déclinaison d'un élément chimique.\\

Si l'on arrivait à mesurer absolument ces ratios, ces ellipses ne seraient que des points. Mais, en physique, la precision absolue n'existe pas. Prenez une règle et mesurez le côté d'une table dix fois au millimetre près. Vous n'obtiendrez jamais la même mesure. Vous pouvez par contre, grâces aux statistiques, determiner une fourchette dans laquelle se trouve la vrai longueur de la table. On fait pareil pour les ratios. On détermine pour les deux une zones dans lesquelles leurs valeurs doit se trouver. Le mélange de ces deux zones, prenant en compte la correlation des deux ratios donne l'ellipse.

Puisque les scientifiques font ces mesures sur plusieurs zircons pour une seule couche terrestre, ils doivent pouvoir mettre toutes ces mesures sur le même schema.

%--------------INSERER ICI GRAPHIQUE AVEC DES CARRÉS. DON'T FORGET--------------



\subsection{Remplacer \textit{Isotop}}
